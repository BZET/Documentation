\documentclass[13pt]{article}

\usepackage{siunitx} % typesets numbers with units very nicely
\usepackage{enumerate} % allows us to customize our lists
\usepackage[brazilian]{babel}
\usepackage[utf8]{inputenc}
\usepackage[T1]{fontenc}

\begin{document}

\title{Sistema de Auxilio � localiza��o de usu�rios com Tecnologia Indoor Wireless Navigation}
\author{Bruna de S� Tavares, José Lucas Araújo, Wilton Sapia Dantas}
\date{\today}
\maketitle  

\section*{Introdu��o}

\section*{Escopo}
	O projeto será dividido em dois módulos: o aplicativo android e o servidor.Estes estarão relacionados através da internet.
\section*{Glossário}

\section*{Refer�ncias}

\section*{Oportunidades de Neg�cio}
	Esse projeto tem como oportunidade de negócio o mapeamento de eventos, locais públicos como: shoppings, aeroportos e terminais de viação.

\section*{Descri�ao dos Stakeholders}
	-Instituto Mauá de Tecnologia \\
	-Funcionários \\
	-Alunos \\
	-Visitantes \\
	-Administradores(nós)\\
	-Linktel \\
	-Evento Eureka \\
	-Evento Mauá Hand-On \\
	-Semana de Engenharia \\
	-iDocent (projeto de referência) \\ 
	
\section*{Ambiente atual do cliente}
\section*{Módulos do Sistema}
\section*{Precedência e Prioridades}
\section*{Requisitos Funcionais}
\section*{Requitos Não Funcionais}
\section*{Restrições}
\section*{Ambiente Operacional}
Java Android, SqLite, MySql, PHP, Apache server, HTML/CSS, Javascript, Ajax.
\section*{Visão Geral do Sistema - Modelo Conceitual}


 \end{document}